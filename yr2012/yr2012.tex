\documentclass{beamer}

\usepackage{fancyvrb}
\usepackage{color}
\usepackage[ascii]{inputenc}

\makeatletter
\def\PY@reset{\let\PY@it=\relax \let\PY@bf=\relax%
    \let\PY@ul=\relax \let\PY@tc=\relax%
    \let\PY@bc=\relax \let\PY@ff=\relax}
\def\PY@tok#1{\csname PY@tok@#1\endcsname}
\def\PY@toks#1+{\ifx\relax#1\empty\else%
    \PY@tok{#1}\expandafter\PY@toks\fi}
\def\PY@do#1{\PY@bc{\PY@tc{\PY@ul{%
    \PY@it{\PY@bf{\PY@ff{#1}}}}}}}
\def\PY#1#2{\PY@reset\PY@toks#1+\relax+\PY@do{#2}}

\expandafter\def\csname PY@tok@gd\endcsname{\def\PY@tc##1{\textcolor[rgb]{0.63,0.00,0.00}{##1}}}
\expandafter\def\csname PY@tok@gu\endcsname{\let\PY@bf=\textbf\def\PY@tc##1{\textcolor[rgb]{0.50,0.00,0.50}{##1}}}
\expandafter\def\csname PY@tok@gt\endcsname{\def\PY@tc##1{\textcolor[rgb]{0.00,0.25,0.82}{##1}}}
\expandafter\def\csname PY@tok@gs\endcsname{\let\PY@bf=\textbf}
\expandafter\def\csname PY@tok@gr\endcsname{\def\PY@tc##1{\textcolor[rgb]{1.00,0.00,0.00}{##1}}}
\expandafter\def\csname PY@tok@cm\endcsname{\let\PY@it=\textit\def\PY@tc##1{\textcolor[rgb]{0.25,0.50,0.50}{##1}}}
\expandafter\def\csname PY@tok@vg\endcsname{\def\PY@tc##1{\textcolor[rgb]{0.10,0.09,0.49}{##1}}}
\expandafter\def\csname PY@tok@m\endcsname{\def\PY@tc##1{\textcolor[rgb]{0.40,0.40,0.40}{##1}}}
\expandafter\def\csname PY@tok@mh\endcsname{\def\PY@tc##1{\textcolor[rgb]{0.40,0.40,0.40}{##1}}}
\expandafter\def\csname PY@tok@go\endcsname{\def\PY@tc##1{\textcolor[rgb]{0.50,0.50,0.50}{##1}}}
\expandafter\def\csname PY@tok@ge\endcsname{\let\PY@it=\textit}
\expandafter\def\csname PY@tok@vc\endcsname{\def\PY@tc##1{\textcolor[rgb]{0.10,0.09,0.49}{##1}}}
\expandafter\def\csname PY@tok@il\endcsname{\def\PY@tc##1{\textcolor[rgb]{0.40,0.40,0.40}{##1}}}
\expandafter\def\csname PY@tok@cs\endcsname{\let\PY@it=\textit\def\PY@tc##1{\textcolor[rgb]{0.25,0.50,0.50}{##1}}}
\expandafter\def\csname PY@tok@cp\endcsname{\def\PY@tc##1{\textcolor[rgb]{0.74,0.48,0.00}{##1}}}
\expandafter\def\csname PY@tok@gi\endcsname{\def\PY@tc##1{\textcolor[rgb]{0.00,0.63,0.00}{##1}}}
\expandafter\def\csname PY@tok@gh\endcsname{\let\PY@bf=\textbf\def\PY@tc##1{\textcolor[rgb]{0.00,0.00,0.50}{##1}}}
\expandafter\def\csname PY@tok@ni\endcsname{\let\PY@bf=\textbf\def\PY@tc##1{\textcolor[rgb]{0.60,0.60,0.60}{##1}}}
\expandafter\def\csname PY@tok@nl\endcsname{\def\PY@tc##1{\textcolor[rgb]{0.63,0.63,0.00}{##1}}}
\expandafter\def\csname PY@tok@nn\endcsname{\let\PY@bf=\textbf\def\PY@tc##1{\textcolor[rgb]{0.00,0.00,1.00}{##1}}}
\expandafter\def\csname PY@tok@no\endcsname{\def\PY@tc##1{\textcolor[rgb]{0.53,0.00,0.00}{##1}}}
\expandafter\def\csname PY@tok@na\endcsname{\def\PY@tc##1{\textcolor[rgb]{0.49,0.56,0.16}{##1}}}
\expandafter\def\csname PY@tok@nb\endcsname{\def\PY@tc##1{\textcolor[rgb]{0.00,0.50,0.00}{##1}}}
\expandafter\def\csname PY@tok@nc\endcsname{\let\PY@bf=\textbf\def\PY@tc##1{\textcolor[rgb]{0.00,0.00,1.00}{##1}}}
\expandafter\def\csname PY@tok@nd\endcsname{\def\PY@tc##1{\textcolor[rgb]{0.67,0.13,1.00}{##1}}}
\expandafter\def\csname PY@tok@ne\endcsname{\let\PY@bf=\textbf\def\PY@tc##1{\textcolor[rgb]{0.82,0.25,0.23}{##1}}}
\expandafter\def\csname PY@tok@nf\endcsname{\def\PY@tc##1{\textcolor[rgb]{0.00,0.00,1.00}{##1}}}
\expandafter\def\csname PY@tok@si\endcsname{\let\PY@bf=\textbf\def\PY@tc##1{\textcolor[rgb]{0.73,0.40,0.53}{##1}}}
\expandafter\def\csname PY@tok@s2\endcsname{\def\PY@tc##1{\textcolor[rgb]{0.73,0.13,0.13}{##1}}}
\expandafter\def\csname PY@tok@vi\endcsname{\def\PY@tc##1{\textcolor[rgb]{0.10,0.09,0.49}{##1}}}
\expandafter\def\csname PY@tok@nt\endcsname{\let\PY@bf=\textbf\def\PY@tc##1{\textcolor[rgb]{0.00,0.50,0.00}{##1}}}
\expandafter\def\csname PY@tok@nv\endcsname{\def\PY@tc##1{\textcolor[rgb]{0.10,0.09,0.49}{##1}}}
\expandafter\def\csname PY@tok@s1\endcsname{\def\PY@tc##1{\textcolor[rgb]{0.73,0.13,0.13}{##1}}}
\expandafter\def\csname PY@tok@sh\endcsname{\def\PY@tc##1{\textcolor[rgb]{0.73,0.13,0.13}{##1}}}
\expandafter\def\csname PY@tok@sc\endcsname{\def\PY@tc##1{\textcolor[rgb]{0.73,0.13,0.13}{##1}}}
\expandafter\def\csname PY@tok@sx\endcsname{\def\PY@tc##1{\textcolor[rgb]{0.00,0.50,0.00}{##1}}}
\expandafter\def\csname PY@tok@bp\endcsname{\def\PY@tc##1{\textcolor[rgb]{0.00,0.50,0.00}{##1}}}
\expandafter\def\csname PY@tok@c1\endcsname{\let\PY@it=\textit\def\PY@tc##1{\textcolor[rgb]{0.25,0.50,0.50}{##1}}}
\expandafter\def\csname PY@tok@kc\endcsname{\let\PY@bf=\textbf\def\PY@tc##1{\textcolor[rgb]{0.00,0.50,0.00}{##1}}}
\expandafter\def\csname PY@tok@c\endcsname{\let\PY@it=\textit\def\PY@tc##1{\textcolor[rgb]{0.25,0.50,0.50}{##1}}}
\expandafter\def\csname PY@tok@mf\endcsname{\def\PY@tc##1{\textcolor[rgb]{0.40,0.40,0.40}{##1}}}
\expandafter\def\csname PY@tok@err\endcsname{\def\PY@bc##1{\setlength{\fboxsep}{0pt}\fcolorbox[rgb]{1.00,0.00,0.00}{1,1,1}{\strut ##1}}}
\expandafter\def\csname PY@tok@kd\endcsname{\let\PY@bf=\textbf\def\PY@tc##1{\textcolor[rgb]{0.00,0.50,0.00}{##1}}}
\expandafter\def\csname PY@tok@ss\endcsname{\def\PY@tc##1{\textcolor[rgb]{0.10,0.09,0.49}{##1}}}
\expandafter\def\csname PY@tok@sr\endcsname{\def\PY@tc##1{\textcolor[rgb]{0.73,0.40,0.53}{##1}}}
\expandafter\def\csname PY@tok@mo\endcsname{\def\PY@tc##1{\textcolor[rgb]{0.40,0.40,0.40}{##1}}}
\expandafter\def\csname PY@tok@kn\endcsname{\let\PY@bf=\textbf\def\PY@tc##1{\textcolor[rgb]{0.00,0.50,0.00}{##1}}}
\expandafter\def\csname PY@tok@mi\endcsname{\def\PY@tc##1{\textcolor[rgb]{0.40,0.40,0.40}{##1}}}
\expandafter\def\csname PY@tok@gp\endcsname{\let\PY@bf=\textbf\def\PY@tc##1{\textcolor[rgb]{0.00,0.00,0.50}{##1}}}
\expandafter\def\csname PY@tok@o\endcsname{\def\PY@tc##1{\textcolor[rgb]{0.40,0.40,0.40}{##1}}}
\expandafter\def\csname PY@tok@kr\endcsname{\let\PY@bf=\textbf\def\PY@tc##1{\textcolor[rgb]{0.00,0.50,0.00}{##1}}}
\expandafter\def\csname PY@tok@s\endcsname{\def\PY@tc##1{\textcolor[rgb]{0.73,0.13,0.13}{##1}}}
\expandafter\def\csname PY@tok@kp\endcsname{\def\PY@tc##1{\textcolor[rgb]{0.00,0.50,0.00}{##1}}}
\expandafter\def\csname PY@tok@w\endcsname{\def\PY@tc##1{\textcolor[rgb]{0.73,0.73,0.73}{##1}}}
\expandafter\def\csname PY@tok@kt\endcsname{\def\PY@tc##1{\textcolor[rgb]{0.69,0.00,0.25}{##1}}}
\expandafter\def\csname PY@tok@ow\endcsname{\let\PY@bf=\textbf\def\PY@tc##1{\textcolor[rgb]{0.67,0.13,1.00}{##1}}}
\expandafter\def\csname PY@tok@sb\endcsname{\def\PY@tc##1{\textcolor[rgb]{0.73,0.13,0.13}{##1}}}
\expandafter\def\csname PY@tok@k\endcsname{\let\PY@bf=\textbf\def\PY@tc##1{\textcolor[rgb]{0.00,0.50,0.00}{##1}}}
\expandafter\def\csname PY@tok@se\endcsname{\let\PY@bf=\textbf\def\PY@tc##1{\textcolor[rgb]{0.73,0.40,0.13}{##1}}}
\expandafter\def\csname PY@tok@sd\endcsname{\let\PY@it=\textit\def\PY@tc##1{\textcolor[rgb]{0.73,0.13,0.13}{##1}}}

\def\PYZbs{\char`\\}
\def\PYZus{\char`\_}
\def\PYZob{\char`\{}
\def\PYZcb{\char`\}}
\def\PYZca{\char`\^}
\def\PYZam{\char`\&}
\def\PYZlt{\char`\<}
\def\PYZgt{\char`\>}
\def\PYZsh{\char`\#}
\def\PYZpc{\char`\%}
\def\PYZdl{\char`\$}
\def\PYZti{\char`\~}
% for compatibility with earlier versions
\def\PYZat{@}
\def\PYZlb{[}
\def\PYZrb{]}
\makeatother


\usetheme{Warsaw}
\usecolortheme{beetle}
\title{Perl is for pwn!}
\author{Sergey Romanov}
\date{YAPC::Russia 2012}

\begin{document}

\frame{\titlepage}

\frame
{
\frametitle{Hello}
\begin{itemize}
\item Sergey Romanov (sromanov on irc.perl.org)
\item Do Perl for fun (also, for living)
\item Like alpacas
\end{itemize}
\begin{figure}
\includegraphics[width=2.5in,height=1.5in]{pics/alpacas.jpg}
\end{figure}
}

\section[CTF]{}

\subsection{What's it all about}
\frame
{
\frametitle{CTF}
\begin{itemize}
\item Capture the Flag (CTF) is a computer security wargame
\item CTF was popularized by the hacker conference DEFCON
\item Two basic types of competition
\end{itemize}
}

\subsection{Task-based CTF}
\frame
{
\frametitle{Find the key}
\begin{itemize}
\item Teams should solve tasks get points
\item Different categories: web, reverse, packet analysis, admin, ctb (crack-the-box), cryptography etc etc
\item Quals are usually task-based
\end{itemize}
}

\subsection{"Classic" CTF}
\frame
{
\frametitle{Steal the flag}
\begin{figure}
\includegraphics[width=3in,height=2in]{pics/network.png}
\end{figure}
}

\subsection{Okay, we get it}
\frame
{
\frametitle{How about Perl?}
\begin{itemize}
\item<1-> Perl is used during CTF game \_heavily\_
\item<2-> Just like any other modern, popular and convinient tool :)
\item<3-> But we'll concentrate on Perl
\end{itemize}
}

\section{Participant side}

\subsection{Routine tools}
\frame
{
\frametitle{CPAN \& beyond}
\begin{itemize}
\item<1-> /usr/bin/lwp-*
\item<1-> /usr/bin/md5pass
\item<2-> find out yours: \textbf{grep '/usr/bin/perl' /usr/bin/*}
\end{itemize}
}

\frame
{
\frametitle{Gort, Klaatu barada...}
Nikto2
\begin{figure}
\includegraphics[width=1in,height=2in]{pics/nikto.png}
\end{figure}
}

\frame
{
\frametitle{Nikto2}
\begin{itemize}
\item Web server scanner
\item Based on libwhisker2 by rain forest puppy (rfp)
\end{itemize}
}

\subsection{Flag poster}
\frame
{
\frametitle{Exploitfarm}
\begin{itemize}
\item Written at Hackerdom (USU, Ekaterinburg)
\item Automate process of collecting flags and submitting them to jury check system.
\end{itemize}
}

\section{Organizer side}
\subsection{Example from RuCTF 2012 Quals}
\frame[containsverbatim]
{
\frametitle{Demo}
\footnotesize{
\begin{Verbatim}[commandchars=\\\{\}]
\PY{k}{sub }\PY{n+nf}{f}\PY{p}{(@d)\PYZob{}}
 \PY{k}{return} \PY{l+m+mi}{0} \PY{k}{unless} \PY{n+nv}{@}\PY{n+nv}{d}\PY{p}{;}
 \PY{k}{my} \PY{n+nv}{\PYZdl{}}\PY{n+nv}{n} \PY{o}{=} \PY{n+nv}{@}\PY{n+nv}{d}\PY{o}{.}\PY{n}{elems}\PY{p}{;}

 \PY{k}{my} \PY{n+nv}{@}\PY{n+nv}{p}\PY{p}{;}
 \PY{n+nb}{push} \PY{n+nv}{@}\PY{n+nv}{p}\PY{p}{,} \PY{p}{[}\PY{l+m+mh}{0x100500} \PY{n}{xx} \PY{n+nv}{\PYZdl{}}\PY{n+nv}{n}\PY{p}{]} \PY{k}{for} \PY{l+m+mi}{0}\PY{o}{..}\PY{o}{\PYZca{}}\PY{l+m+mi}{1}\PY{o}{+}\PY{o}{\PYZlt{}}\PY{n+nv}{\PYZdl{}}\PY{n+nv}{n}\PY{p}{;}
 \PY{n+nv}{@}\PY{n+nv}{p}\PY{p}{[}\PY{l+m+mi}{0}\PY{p}{]}\PY{p}{[}\PY{l+m+mi}{0}\PY{p}{]}\PY{o}{=}\PY{l+m+mi}{0}\PY{p}{;}

 \PY{k}{return} \PY{p}{[}\PY{n}{min}\PY{p}{]}\PY{n}{gather} \PY{k}{for} \PY{l+m+mi}{1}\PY{p}{,}\PY{o}{*}\PY{o}{+}\PY{l+m+mi}{2}\PY{o}{...}\PY{l+m+mi}{1}\PY{o}{+}\PY{o}{\PYZlt{}}\PY{n+nv}{\PYZdl{}}\PY{n+nv}{n}\PY{o}{-}\PY{l+m+mi}{1} \PY{o}{-}\PY{o}{\PYZgt{}}\PY{n+nv}{\PYZdl{}}\PY{n+nv}{x}\PY{p}{\PYZob{}}
  \PY{k}{for} \PY{p}{(}\PY{l+m+mi}{1}\PY{o}{..}\PY{o}{\PYZca{}}\PY{n+nv}{\PYZdl{}}\PY{n+nv}{n}\PY{p}{)}\PY{o}{.}\PY{n+nb}{grep}\PY{p}{(}\PY{p}{\PYZob{}}\PY{n+nv}{\PYZdl{}}\PY{n+nv}{x}\PY{o}{+}\PY{o}{\PYZam{}}\PY{l+m+mi}{1}\PY{o}{+}\PY{o}{\PYZlt{}}\PY{n+nv}{\PYZdl{}}\PY{n+nv}{x}\PY{p}{\PYZcb{}}\PY{p}{)}\PY{n}{X}\PY{p}{(}\PY{l+m+mi}{0}\PY{o}{..}\PY{o}{\PYZca{}}\PY{n+nv}{\PYZdl{}}\PY{n+nv}{n}\PY{p}{)}\PY{o}{.}\PY{n+nb}{grep}\PY{p}{(}\PY{p}{\PYZob{}}\PY{n+nv}{\PYZdl{}}\PY{n+nv}{x}\PY{o}{+}\PY{o}{\PYZam{}}\PY{l+m+mi}{1}\PY{o}{+}\PY{o}{\PYZlt{}}\PY{n+nv}{\PYZdl{}}\PY{n+nv}{x}\PY{p}{\PYZcb{}}\PY{p}{)} \PY{o}{-}\PY{o}{\PYZgt{}}\PY{n+nv}{\PYZdl{}}\PY{n+nv}{z}\PY{p}{,}\PY{n+nv}{\PYZdl{}}\PY{n+nv}{c}\PY{p}{\PYZob{}}
   \PY{n+nv}{@}\PY{n+nv}{p}\PY{p}{[}\PY{n+nv}{\PYZdl{}}\PY{n+nv}{x}\PY{p}{]}\PY{p}{[}\PY{n+nv}{\PYZdl{}}\PY{n+nv}{z}\PY{p}{]}\PY{o}{=}\PY{p}{[}\PY{n}{min}\PY{p}{]}\PY{n+nv}{@}\PY{n+nv}{p}\PY{p}{[}\PY{n+nv}{\PYZdl{}}\PY{n+nv}{x}\PY{p}{]}\PY{p}{[}\PY{n+nv}{\PYZdl{}}\PY{n+nv}{z}\PY{p}{]}\PY{p}{,}\PY{n+nv}{@}\PY{n+nv}{p}\PY{p}{[}\PY{n+nv}{\PYZdl{}}\PY{n+nv}{x}\PY{o}{+}\PY{o}{\PYZca{}}\PY{l+m+mi}{1}\PY{o}{+}\PY{o}{\PYZlt{}}\PY{n+nv}{\PYZdl{}}\PY{n+nv}{z}\PY{p}{]}\PY{p}{[}\PY{n+nv}{\PYZdl{}}\PY{n+nv}{c}\PY{p}{]}\PY{p}{,}\PY{n+nv}{@}\PY{n+nv}{d}\PY{p}{[}\PY{n+nv}{\PYZdl{}}\PY{n+nv}{c}\PY{p}{]}\PY{p}{[}\PY{n+nv}{\PYZdl{}}\PY{n+nv}{z}\PY{p}{]}
  \PY{p}{\PYZcb{}}
  \PY{n}{take} \PY{n+nv}{@}\PY{n+nv}{p}\PY{p}{[}\PY{l+m+mi}{1}\PY{o}{+}\PY{o}{\PYZlt{}}\PY{n+nv}{\PYZdl{}}\PY{n+nv}{n}\PY{o}{-}\PY{l+m+mi}{1}\PY{p}{]}\PY{p}{[}\PY{n+nv}{\PYZdl{}}\PY{n+nv}{\PYZus{}}\PY{p}{]}\PY{o}{+}\PY{n+nv}{@}\PY{n+nv}{d}\PY{p}{[}\PY{n+nv}{\PYZdl{}}\PY{n+nv}{\PYZus{}}\PY{p}{]}\PY{p}{[}\PY{l+m+mi}{0}\PY{p}{]} \PY{k}{for} \PY{o}{\PYZca{}}\PY{n+nv}{\PYZdl{}}\PY{n+nv}{n}
 \PY{p}{\PYZcb{}}
\PY{p}{\PYZcb{}}
\end{Verbatim}
}
}

\subsection{Services}
\frame
{
\frametitle{Demo}
}

\subsection{Check system}
\frame
{
\frametitle{Complex system for CTF-style contests}
\begin{itemize}
\item<1-> Written by Lexi Pimenidis, RWTH Aachen
\item<1-> Gameserver, the Submitserver, and the Scoreserver
\item<1-> Was used at CIPHER, op3n, UralCTF	etc
\end{itemize}
}

\frame
{
\frametitle{Links}
\item Nikto2: http://cirt.net/nikto2
\item Exploitfarm: http://code.google.com/p/exploitfarm/
\item CIPHER Gameserver: http://www.cipher-ctf.org/Gameserver.php
\item This talk: https://speakerdeck.com/u/sromanov/p/perl-is-for-pwn
\item Twitter: @SR0MAN0V (yes, zeros instead of "O"s)
}

\frame
{
\frametitle{Thank you!}
\begin{figure}
\includegraphics[width=3in,height=2in]{pics/aybabtu.png}
\end{figure}

PS: DEFCON XX Quals start 2 Jun 2012! Join!
}

\end{document}
